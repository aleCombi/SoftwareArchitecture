\subsection{Decision 13: Serverless vs. Servers for Calculations}

\subsection*{Status}
Open

\subsection*{Architectural Summary}
This decision examines the computational approach for the TrIP system, specifically whether to adopt a serverless architecture or to maintain dedicated servers for processing calculations related to ticketing, route optimization, and other system functionalities.

\subsection*{Concern}
The primary concern is to select a computational architecture that balances scalability, cost, performance, and data privacy for processing passenger data and transactional information.

\subsection*{Context}
The computational backbone of the TrIP system must handle variable workloads efficiently, especially during peak hours when route calculations and payment processing are at their highest demand. Additionally, the system must maintain data privacy and adhere to regulatory compliance.

\subsection*{Criteria}
\begin{itemize}
    \item Scalability to handle peak and off-peak loads.
    \item Cost-effectiveness, including operational and maintenance costs.
    \item Performance in terms of latency and throughput.
    \item Data privacy and control.
    \item Compliance with data protection and privacy laws.
    \item Ease of maintenance and updates.
\end{itemize}

\subsection*{Option 1: Serverless}
Adopting a serverless architecture where the service provider dynamically manages the allocation of machine resources.
\subsubsection*{Pros}
\begin{itemize}
    \item Cost efficiency during low usage.
    \item No need for server maintenance.
    \item Automatic scaling.
\end{itemize}
\subsubsection*{Cons}
\begin{itemize}
    \item Potential for increased latency.
    \item Less control over data.
    \item Possible security concerns.
\end{itemize}

\subsection*{Option 2: Dedicated Servers}
Using dedicated servers, either on-premises or hosted, to handle all computations.
\subsubsection*{Pros}
\begin{itemize}
    \item Greater control over data.
    \item Potentially better performance.
    \item Consistent availability.
\end{itemize}
\subsubsection*{Cons}
\begin{itemize}
    \item Higher upfront costs.
    \item Requires dedicated IT staff for maintenance.
    \item Might be underutilized during off-peak times.
\end{itemize}

\subsection*{Option 3: Hybrid Approach}
Implementing a hybrid system that uses a combination of serverless architecture for less sensitive and highly variable workloads, and dedicated servers for more predictable workloads and data-intensive tasks requiring stringent privacy controls.
\subsubsection*{Pros}
\begin{itemize}
    \item Balances the benefits of both serverless and dedicated servers.
    \item Provides scalability while maintaining data privacy for sensitive operations.
\end{itemize}
\subsubsection*{Cons}
\begin{itemize}
    \item Increased complexity in managing two different environments.
    \item Potential for higher operational costs.
\end{itemize}

\subsection*{Decision}
The decision will be based on a cost-benefit analysis of each option, considering the trade-offs between control, cost, and scalability. The chosen approach must align with the overall system requirements for performance, privacy, and regulatory compliance.

\subsection*{Consequences}
\textbf{Positive Consequences:}
% \begin{itemize}
% \end{itemize}
\textbf{Negative Consequences:}
% \begin{itemize}
% \end{itemize}