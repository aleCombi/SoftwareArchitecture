\subsection{Decision 14: Revenue Division Strategies}

\subsection*{Status}
To be decided.

\subsection*{Architectural Summary}

\subsection*{Concern}
The main concern of the users in the context of this decision are descibed by the following user stories: 
\begin{itemize}
    \item User Story 21: Tycoons require accurate revenue tracking.
    \item User Story 23: Tycoons demand minimal disruption to existing infrastructure.
    \item User Story 24: Tycoons need analytics and insights for business decisions.
\end{itemize}

\subsection*{Context}
The revenue division strategy impacts the following functional elements and databases within the Train Inter Payment System (TrIP):
\begin{itemize}
    \item Payment Terminals: Interface for collecting revenue data.
    \item Tycoon-Specific Systems: Must integrate with revenue distribution logic.
    \item Databases:
    \begin{itemize}
        \item Ticket Database: Stores ticket sales data.
        \item Payment Database: Tracks completed transactions.
        \item Accounts and Subscriptions Database: Manages passenger subscriptions impacting revenue sharing.
    \end{itemize}
    \item Booking Management Module: Influences revenue calculation based on booked seats.
    \item Account and Subscription Management Module: Central to managing subscription-related revenue data.
\end{itemize}

\subsection*{Criteria}
The goal of this decision is to satisfy the user's concerns while safeguarding the other quality attribute requirements of the system. The main QAs affected by this decision are listed here:
\begin{itemize}
    \item Functional Suitability: Functionally complete, correct, and appropriate method for revenue distribution.
    \item Performance Efficiency: Strategy must not adversely affect system response times or resource utilization.
    \item Compatibility: Must coexist and interoperate with the tycoons' diverse systems and databases.
    \item Security: Ensuring confidentiality and integrity of revenue data.
    \item Maintainability: Ability to analyze, modify, and test revenue division logic as needed.
    \item Reliability: Fault tolerance and availability of the revenue division process.
    \item Flexibility: Adaptability and scalability of the strategy to accommodate new tycoons or changing business models.
\end{itemize}

\subsection*{Options}
\begin{itemize}
    \item \textbf{Tap Cards Onboard Transport Vehicles}: Allocate revenue to the tycoon operating each transport vehicles based on passenger tap card data. This approach directly links revenue to individual train usage.
    \begin{itemize}
        \item \textit{Pros}: Directly correlates revenue with service usage, incentivizing tycoons to improve service quality and increase ridership.
        \item \textit{Cons}: May not fully account for the value of network-wide contributions, such as infrastructure maintenance or off-peak services.
    \end{itemize}
    \item \textbf{Annual Value-Based Share}: Distribute revenue based on a combination of the value of each tycoon's contributions (infrastructure and services) and ticket sales data from the previous year.
    \begin{itemize}
        \item \textit{Pros}: Acknowledges both the operational and capital contributions of tycoons, potentially offering a more balanced revenue share.
        \item \textit{Cons}: Risks disadvantaging new entrants or those expanding services, as it relies on historical data.
    \end{itemize}
    \item \textbf{Negotiated Shares}: Tycoons negotiate revenue shares periodically, based on a set of agreed criteria (e.g., service quality, passenger numbers, network investment).
    \begin{itemize}
        \item \textit{Pros}: Allows for flexibility and adaptability in revenue sharing, can directly address the specific contributions and needs of each tycoon.
        \item \textit{Cons}: Could lead to conflicts or prolonged negotiations, potentially destabilizing the revenue sharing process.
    \end{itemize}
\end{itemize}

\subsection*{Decision}
(Leave this section empty as requested.)

\subsection*{Consequences}
(Leave this section empty as requested.)