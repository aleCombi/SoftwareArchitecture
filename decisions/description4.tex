\section*{Decision 4: How to integrate subscription to route optimization module}

\subsection*{Status}
Review

\subsection*{Architectural Summary}
In developing the TrIP system, we explore subscription model architectures that align railway tycoons' need for flexibility with passengers' demand for simplicity. We assess the trade-offs between unified and tycoon-specific models to enhance both operational autonomy and passenger convenience.

\subsection*{Concern}
A user wants to know the available tickets from A-B. 
Connected user stories: 16

\subsection*{Context}
How the system handles the request by the user of a route. The system has to check the available routes, the prices/time of available routes.

\subsection*{Criteria}
\begin{itemize}
\item Ease of use for the passenger.
\item Multiple route options for the user.
\item The system returns to user usable options based on price/time/availability/subscription.
\item Simple integration with the tycoon systems.
\end{itemize}

\subsection*{Option 1: Terminal communicates with tycoons, our database and a route optimizer module}
We want to structure our system in 4 modules - Terminal, tycoon, database, route optimizer.
Insert figure.

\subsection*{Option 2: Terminal communicates with route optimizer and data management module}
We want to structure our system in 5 modules - Terminal, tycoon, database, database management and route optimizer.
Insert figure.

\subsection*{Decision}
We choose option 2.
Easier integration with tycoons.
Separation of concerns, having additional database management module is more extendible.

\subsection*{Consequences}
\textbf{Positive Consequences:}
\begin{itemize}
\item Simplifies data flow between the trip system and tycoons.
\end{itemize}
\textbf{Negative Consequences:}
\begin{itemize}
\item Creating additional module which requires development.
\end{itemize}
This approach seeks to balance stakeholder interests, with room for iterative refinement as the system evolves.

