\subsection{Decision 9: Account management and authentication}

\subsection*{Status}
Review

\subsection*{Architectural Summary}
\begin{tabular}{|p{3.5cm}|p{10.5cm}|}
    \hline
    \textbf{In the context of} & letting passengers authenticate to their accounts to handle subscriptions, \\
    \hline
    \textbf{Facing} & the passengers' concern for user-friendlyness and the tycoon request for integration between their subscriptions systems, also ensuring sensitive data privacy, \\
    \hline
    \textbf{To achieve} & security, usability for passengers and integrabiliy with tycoons systems, \\
    \hline
    \textbf{We considered} & Option 1: Anonymous card; Option 2: Bank credit card; Option 3: Phone app; Option 4: Nominal card; Option 5: NFC reader/Google Wallet\\
    \hline
    \textbf{And decided for} & a hybrid approach where multiple authentication methods are allowed, both nominal (like a dedicated card) and anomymous (like a fillable card);\\
    \hline
    \textbf{Because} & it ensures high usability for many kinds of passengers, only two readers are needed (NFC and QR) at turnstiles, which keeps costs controlled, security is ensured by tokenization, \\
    \hline
    \textbf{Accepting} & Higer system complexity, more complicated development and maintenance. \\
    \hline
\end{tabular}

\subsection*{Concern}
Ensuring a secure, user-friendly, and efficient process for account creation, management, and authentication for a variety of users, while maintaining data privacy and system integrity.

Related user stories are listed below:
\begin{itemize}
    \item \userStoryOne
    \item \userStoryFour
    \item \userStoryTwelve
    \item \userStoryEighteen
    \item \userStoryTwentySix
\end{itemize}

\subsection*{Context}
The TrIP system requires a flexible and secure method for user authentication that accommodates various levels of user interaction and convenience. The chosen solution must integrate seamlessly with the existing ticketing and payment infrastructure and support a range of devices and technologies used by passengers.

\subsection*{Criteria}
\begin{itemize}
    \item \textit{Security}: Secure storage and handling of personal and financial data.
    \item \textit{Usability}: Ease of account creation and management for passengers.
    \item \textit{Integrability}: Seamless integration with existing turnstile and payment systems.
    \item \textit{Flexibility}: Flexibility to support different methods of authentication, including cards and digital wallets.
    \item \textit{Availability}: High availability and reliability of the authentication system.
\end{itemize}

\subsection*{Option 1: Anonymous card}
\begin{itemize}
    \item \textbf{Pro:} Ensures passenger privacy and quick adoption for casual users.
    \item \textbf{Con:} Limited capabilities for account management and tracking passenger history.
\end{itemize}

\subsection*{Option 2: Bank credit card}
\begin{itemize}
    \item \textbf{Pro:} Streamlines payment process by integrating with existing financial systems.
    \item \textbf{Con:} Relies on external systems, which may pose integration challenges and dependency risks.
\end{itemize}

\subsection*{Option 3: Phone app}
\begin{itemize}
    \item \textbf{Pro:} Provides a versatile platform for account management, payments, and ticket validation via QR codes or NFC.
    \item \textbf{Con:} Requires smartphone access, possibly excluding certain passengers demographics.
\end{itemize}

\subsection*{Option 4: Nominal card}
\begin{itemize}
    \item \textbf{Pro:} Offers a physical, personalized token for account access and ticket validation.
    \item \textbf{Con:} May introduce additional costs for production and distribution of cards.
\end{itemize}

\subsection*{Option 5: NFC reader/Google Wallet}
\begin{itemize}
    \item \textbf{Pro:} Leverages existing NFC technology in smartphones for easy tap-and-go access at turnstiles.
    \item \textbf{Con:} Implementation cost and required NFC-capable turnstiles may be high.
\end{itemize}

\subsection*{Decision}
The decision is to implement a hybrid approach, integrating a phone app (with NFC capabilities and linked to bank accounts) for regular users, along with an anonymous fillable card for tourists and a physical nominal card for those who prefer or require a physical token. The hybrid approach balances convenience with coverage for all passenger types.
A tokenization service can be implemented, so that no sensitive data is stored, but only tokens representing it, which transfer the security risk and the compliance requirements to specialized companies.
This will be explained in more detailed in the Information Viewpoint.

\subsection*{Consequences}
\textbf{Positive Consequences:}
\begin{itemize}
    \item Provides a comprehensive solution covering the needs of various passenger groups, increasing system accessibility and passenger satisfaction.
    \item Enhances security by offering multiple authentication methods, each with its own set of security protocols.
    \item Encourages digital transformation and supports a move towards a more contactless, efficient user experience.
\end{itemize}
\textbf{Negative Consequences:}
\begin{itemize}
    \item May increase complexity and cost of the system, both in development and maintenance.
    \item The need to support multiple authentication methods can complicate infrastructure and operational processes.
    \item Potential resistance from passengers who are less technologically adept or prefer traditional methods.
\end{itemize}
This decision supports the TrIP system's objective to provide a secure, convenient, and inclusive environment for all types of passengers while recognizing the need to manage operational and development complexities.
