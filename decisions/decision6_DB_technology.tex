\subsection{Decision 6: Database technology}

\subsection*{Status}
Accepted. Reviewed after Event 4 (Data Leaks) with higher focus on security. 

\subsection*{Architectural Summary}
\begin{tabular}{|p{3.5cm}|p{10.5cm}|}
    \hline
    \textbf{In the context of} & Choosing appropriate database technology to store information about travels and payments, \\
    \hline
    \textbf{Facing} & The challenge of storing data coming from different tycoons systems in a cost-effective manner, \\
    \hline
    \textbf{To achieve} & Mainteinability, security, cost efficiency, performance and scalability \\
    \hline
    \textbf{We considered} & Option 1: SQL Database (e.g., PostgreSQL); Option 2: NoSQL Database (e.g., MongoDB); \\
    \hline
    \textbf{And decided for} & Option 1: SQL Database (e.g., PostgreSQL) \\
    \hline
    \textbf{Because} & It suits the kind of data we need to store, it is easier to source developers expertise and security is well tested, \\
    \hline
    \textbf{Accepting} & Potential difficulties in horizontal scaling, less flexibility in the data model. \\
    \hline
\end{tabular}

\subsection*{Concern}
The main staleholder's concerns related to this decision are the passengers request to have a good integration among different tycoons subscription, the TrIP owner request to keep maintenance costs low and the data protection required by government and passengers.

Related user stories are listed here:

\begin{itemize}[noitemsep]
    \item \userStoryEighteen,
    \item \userStoryTwentyNine,
    \item \userStoryThirty,
    \item \userStoryThirtyTwo,
\end{itemize}

\subsection*{Context}
In developing the TrIP system we are faced with the decision of choosing an appropriate database technology. This choice hinges on our need to ensure data integrity, support complex queries for transaction processing, and maintain scalability and security.
The database must handle a wide array of data, including user subscriptions, fare transactions, and station and route information, necessitating a robust system that supports complex queries and relational data structuring.
The architecture might include more than one databases, depending on the needs that will arise during later decisions.

\subsection*{Criteria}
\begin{itemize}[noitemsep]
    \item \textit{Cost efficiency} and \textit{mainteinability}.
    \item Comprehensive \textit{security} features to safeguard sensitive data.
    \item Data integrity and transactional consistency for financial transactions.
    \item Ability to support complex queries and relational data models.
    \item \textit{Scalability} to grow with the system's user base and data volume.
    \item \textit{Performance} under varying load conditions.
\end{itemize}

\subsection*{Option 1: SQL Database (e.g., PostgreSQL)}
A relational database model renowned for its strong consistency, ACID (Atomicity, Consistency, Isolation, and Durability) compliance, and the ability to efficiently handle complex queries and data relationships.
\begin{itemize}
    \item \textbf{Pro:} High data integrity and robust support for complex relational data structures.
    \item \textbf{Pro:} We have a highly structured data.
    \item \textbf{Pro:} More developers are familiar with it, more resources on the topic. It has a strong community and over 30 years of active development.
    \item \textbf{Pro:} Good with concurrency.
    \item \textbf{Pro:} Well tested security features like encryption, access control, SQL injection prevention, auditing capabilities.
    \item \textbf{Pro:} Open source and free.
    \item \textbf{Con:} Scalability challenges in horizontally distributed architectures compared to NoSQL options.
    \item \textbf{Con:} Requires accurate upfront planning of the data model due to its structured nature, thereby limiting flexibility.
\end{itemize}

\subsection*{Option 2: NoSQL Database (e.g., MongoDB)}
A distributed database system designed for scalability and flexibility, suitable for handling large volumes of diverse data types.
\begin{itemize}
    \item \textbf{Pro:} Offers superior scalability and flexibility for managing unstructured or semi-structured data.
    \item \textbf{Pro:} Enhances performance for non-structured data.
    \item \textbf{Con:} May compromise transactional integrity and consistency in favor of performance and scalability.
    \item \textbf{Con:} Relatively new, hence less well tested for security concerns.
\end{itemize}

\subsection*{Decision}
After thorough consideration, the decision is to implement an \textbf{SQL database}, specifically PostgreSQL for it being open source, for the TrIP system. This decision is underpinned by the SQL database's unmatched data integrity, support for complex transactions, and relational data modeling capabilities, which are crucial for the financial transactions and data relationships inherent in the TrIP system. Furthermore, train payment data are by nature very structured, don't give much creativity to the passengers, thus a relational database seems a more natural choice. Given the higher familiarity of developers, this choice addresses well the mainteinability required by the TrIP owner.

\subsection*{Consequences}
\textbf{Positive Consequences:}
\begin{itemize}
    \item Ensures high levels of data integrity and transactional consistency, critical for financial data and user subscriptions.
    \item Facilitates complex data queries and relationships, enabling sophisticated data analysis and reporting.
    \item Provides robust security features to protect sensitive data and comply with data protection regulations.
\end{itemize}
\textbf{Negative Consequences:}
\begin{itemize}
    \item May require additional strategies for scaling horizontally, such as implementing read replicas or sharding, to manage large data volumes and high traffic loads effectively.
    \item Could necessitate more intensive resource management and optimization to ensure performance at scale.
    \item A less flexible data model.
\end{itemize}