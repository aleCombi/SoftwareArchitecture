\subsection{Decision 9: Account management and authentication}

\subsection*{Status}
Review

\subsection*{Architectural Summary}
The system will offer multiple options for account management and authentication to accommodate the preferences and needs of diverse user groups, including commuters, tourists, and casual riders. This strategy aims to streamline the user experience while enhancing security and operational efficiency.

\subsection*{Concern}
Ensuring a secure, user-friendly, and efficient process for account creation, management, and authentication for a variety of users, while maintaining data privacy and system integrity.

\subsection*{Context}
The TrIP system requires a flexible and secure method for user authentication that accommodates various levels of user interaction and convenience. The chosen solution must integrate seamlessly with the existing ticketing and payment infrastructure and support a range of devices and technologies used by passengers.

\subsection*{Criteria}
\begin{itemize}
    \item Secure storage and handling of personal and financial data.
    \item Ease of account creation and management for users.
    \item Seamless integration with existing turnstile and payment systems.
    \item Flexibility to support different methods of authentication, including cards and digital wallets.
    \item High availability and reliability of the authentication system.
\end{itemize}

\subsection*{Option 1: Anonymous card}
\begin{itemize}
    \item \textbf{Pro:} Ensures user privacy and quick adoption for casual users.
    \item \textbf{Con:} Limited capabilities for account management and tracking user history.
\end{itemize}

\subsection*{Option 2: Bank credit card}
\begin{itemize}
    \item \textbf{Pro:} Streamlines payment process by integrating with existing financial systems.
    \item \textbf{Con:} Relies on external systems, which may pose integration challenges and dependency risks.
\end{itemize}

\subsection*{Option 3: Phone app}
\begin{itemize}
    \item \textbf{Pro:} Provides a versatile platform for account management, payments, and ticket validation via QR codes or NFC.
    \item \textbf{Con:} Requires smartphone access, possibly excluding certain user demographics.
\end{itemize}

\subsection*{Option 4: Nominal card}
\begin{itemize}
    \item \textbf{Pro:} Offers a physical, personalized token for account access and ticket validation.
    \item \textbf{Con:} May introduce additional costs for production and distribution of cards.
\end{itemize}

\subsection*{Option 5: NFC reader/Google Wallet}
\begin{itemize}
    \item \textbf{Pro:} Leverages existing NFC technology in smartphones for easy tap-and-go access at turnstiles.
    \item \textbf{Con:} Implementation cost and required NFC-capable turnstiles may be high.
\end{itemize}

\subsection*{Decision}
The decision is to implement a hybrid approach, integrating a phone app (with NFC capabilities and linked to bank accounts) for regular users, along with an anonymous fillable card for tourists and a physical nominal card for those who prefer or require a physical token. The hybrid approach balances convenience with coverage for all user types.

\subsection*{Consequences}
\textbf{Positive Consequences:}
\begin{itemize}
    \item Provides a comprehensive solution covering the needs of various user groups, increasing system accessibility and user satisfaction.
    \item Enhances security by offering multiple authentication methods, each with its own set of security protocols.
    \item Encourages digital transformation and supports a move towards a more contactless, efficient user experience.
\end{itemize}
\textbf{Negative Consequences:}
\begin{itemize}
    \item May increase complexity and cost of the system, both in development and maintenance.
    \item The need to support multiple authentication methods can complicate infrastructure and operational processes.
    \item Potential resistance from users who are less technologically adept or prefer traditional methods.
\end{itemize}
This decision supports the TRIP system's objective to provide a secure, convenient, and inclusive environment for all types of users while recognizing the need to manage operational and development complexities.
