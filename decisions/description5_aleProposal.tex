\section*{Decision 5: How to handle quickly filling trains?}

\subsection*{Status}
Review

\subsection*{Architectural Summary}
QA scenario (availability): When a route is requested by a passenger to a terminal, the terminal verifies that the trains arent fully booked and then proceeds to
ask for the payment.
It could happen that at the same time another passenger buys the last ticket. This is relevant expecially in peak hours, which are typically important for commuters and students.

\subsection*{Concern}
Passenger wants minimum usability and availability. Also, we (and the tycoons) don't want to have customer service issues.
Connected user stories: 15 (passenger don't want to pay for unavailable routes)

\subsection*{Context}


\subsection*{Criteria}
\begin{itemize}
\item Not too many requests to tycoon systems.
\item Performance, to avoid passenger have a bad user experience.
\item Don't allow to pay two users for the same spot.
\end{itemize}

\subsection*{Option 1: }
We should maintain a database where info about scheduled trains is stored. 
This database should be updated by the terminal when a ticket has been bought.
This database should ask periodically to the tycoon systems updates on schedules.
When the user select that they want to pay for a specific ticket, that ticket should be locked, so that no one else can buy it. 
If the payment is not ultimated, it can be unlocked.
Note that it can still happend that due to maintenance, a train is cancelled last minute. 
What happens with the customer service? Should we add it to the context view? They will for sure need info from us, like who paid and how much. 
These should be two future decisions (how to deal with customer service in case of train disruptions, how to keep up to date about disruptions).

\subsection*{Option 2: }

\subsection*{Decision}

\subsection*{Consequences}
\textbf{Positive Consequences:}
\begin{itemize}
\end{itemize}
\textbf{Negative Consequences:}
\begin{itemize}
\end{itemize}