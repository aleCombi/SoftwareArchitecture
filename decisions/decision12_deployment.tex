\subsection{Decision 12: Serverless vs. Servers for Calculations}

\subsection*{Status}
Open

\subsection*{Architectural Summary}
\begin{tabular}{|p{3.5cm}|p{10.5cm}|}
    \hline
    \textbf{In the context of} & choosing to deploy TrIP system on cloud or on physical servers, \\
    \hline
    \textbf{Facing} & cost-effectiveness required by the TrIP owner, scalability and performance needs, \\
    \hline
    \textbf{To achieve} & high availability of the system together with security of passengers data, \\
    \hline
    \textbf{We considered} & Option 1: Serverless architecture; Option 2: Dedicated Servers; Option 3: Hybrid Approach\\
    \hline
    \textbf{And decided for} & Option 3: Hybrid Approach \\
    \hline
    \textbf{Because} & it ensures security where it is needed and cost-effectiveness for the rest of the system, \\
    \hline
    \textbf{Accepting} & Higer operation costs and more involved initial setup and mainentance. \\
    \hline
\end{tabular}

\subsection*{Concern}
The primary concern is to select a computational architecture that balances scalability, cost, performance, and data privacy for processing passenger data and transactional information.

Related user stories are listed here:
\begin{itemize}
    \item \userStoryTwentyNine
    \item \userStoryThirty
    \item \userStoryThirtyOne
    \item \userStoryEighteen
\end{itemize}

\subsection*{Context}
The computational backbone of the TrIP system must handle variable workloads efficiently, especially during peak hours when route calculations and payment processing are at their highest demand. Additionally, the system must maintain data privacy and adhere to regulatory compliance.

\subsection*{Criteria}
\begin{itemize}
    \item \textit{Scalability} to handle peak and off-peak loads.
    \item \textit{Cost efficiency}, including operational and maintenance costs.
    \item \textit{Performance} in terms of latency and throughput.
    \item \textit{Security}: Data privacy and control.
    \item Compliance with data protection and privacy laws.
    \item \textit{Maintainability}: Ease of maintenance and updates.
\end{itemize}

\subsection*{Option 1: Serverless architecture}
Adopting a serverless architecture where the service provider dynamically manages the allocation of machine resources.
\subsubsection*{Pros}
\begin{itemize}
    \item Cost efficiency during low usage.
    \item No need for server maintenance.
    \item Automatic scaling.
\end{itemize}
\subsubsection*{Cons}
\begin{itemize}
    \item Potential for increased latency.
    \item Less control over data.
    \item Possible security concerns.
\end{itemize}

\subsection*{Option 2: Dedicated Servers}
Using dedicated servers, either on-premises or hosted, to handle all computations.
\subsubsection*{Pros}
\begin{itemize}
    \item Greater control over data.
    \item Potentially better performance.
    \item Consistent availability.
\end{itemize}
\subsubsection*{Cons}
\begin{itemize}
    \item Higher upfront costs.
    \item Requires dedicated IT staff for maintenance.
    \item Might be underutilized during off-peak times.
\end{itemize}

\subsection*{Option 3: Hybrid Approach}
Implementing a hybrid system that uses a combination of serverless architecture for less sensitive and highly variable workloads, and dedicated servers for more predictable workloads and data-intensive tasks requiring stringent privacy controls.
\subsubsection*{Pros}
\begin{itemize}
    \item Balances the benefits of both serverless and dedicated servers.
    \item Provides scalability while maintaining data privacy for sensitive operations.
\end{itemize}
\subsubsection*{Cons}
\begin{itemize}
    \item Increased complexity in managing two different environments.
    \item Potential for higher operational costs.
\end{itemize}

\subsection*{Decision}
As the focus on security have increased after Event 4 and the TrIP owner is more prone on spending, we pick option 3. This way, all the privacy-critical operations and data storages will be done in private clouds, while operations and storages of public data (such as the timetables),
will be on public data, guaranteeing cost-effectiveness and less maintenance concerns.
We choose to store and operate with accounts in the system identifying them via an account ID. The mapping between this account IDs and the account private information (such as name, contacts, etc.) should be stored in a in-house private server.
Communication with this database need to be encrypted to guarantee compliance with GDPR, write access to it can only be done by passengers while managing their accounts and read writes can be guaranteed only to officials verifying tickets onboard trains.
Note that credit card data are stored in the system as tockens, so they are not considered sensitive information.

\subsection*{Consequences}
\textbf{Positive Consequences:}
\begin{itemize}
    \item \textbf{Scalability:} Efficient adaptation to variable workloads with serverless, and reliable performance for constant workloads on dedicated servers.
    \item \textbf{Cost-Effectiveness:} Reduced operational costs through serverless computing for dynamic workloads.
    \item \textbf{Performance:} Optimized performance for data-intensive tasks on dedicated servers.
    \item \textbf{Security and Data Privacy:} Enhanced control over sensitive data processing and storage.
\end{itemize}

\textbf{Negative Consequences:}
\begin{itemize}
    \item \textbf{Complexity:} Increased management complexity blending serverless and server environments.
    \item \textbf{Operational Costs:} Potential rise in costs associated with maintaining dedicated server infrastructure.
    \item \textbf{Integration Challenges:} Difficulties in seamless operation between serverless and server-based components.
    \item \textbf{Dependency on Cloud Providers:} Reliance on third-party services for serverless components.
\end{itemize}
