\subsection{Decision 11: Disruptions and Route Updates}

\subsection*{Status}
Open

\subsection*{Architectural Summary}

\begin{tabular}{|p{3.5cm}|p{10.5cm}|}
    \hline
    \textbf{In the context of} & Managing disruptions effectively within the TrIP system. \\
    \hline
    \textbf{Facing} & The need to minimize inconvenience for passengers during disruptions. \\
    \hline
    \textbf{To achieve} & A balance between rapid response and clear communication with passengers. \\
    \hline
    \textbf{We considered} & \begin{tabular}{@{}l@{}}1. Real-Time Alert System, \\ 2. Manual Intervention Protocol, \\ 3. Threshold-Based Re-Optimization, \\ 4. Continuous Optimization.\end{tabular} \\
    \hline
    \textbf{And decided for} & Option 3: Threshold-Based Re-Optimization. \\
    \hline
    \textbf{Because} & It optimally balances operational efficiency and passenger communication, focusing resources on significant disruptions. \\
    \hline
    \textbf{Accepting} & The potential oversight of minor delays and the reliance on terminal-based information. \\
    \hline
\end{tabular}


\subsection*{Concern}
The main concern is to ensure minimal inconvenience to passengers during train disruptions while maintaining transparent communication and providing alternative solutions.

Related user stories are listed below:
\begin{itemize}[noitemsep]
    \item \userStoryTwentySix
\end{itemize}

\subsection*{Context}
Train disruptions can occur due to various reasons such as maintenance issues, accidents, or natural events. The system needs to be able to quickly respond to such incidents, inform affected passengers, and offer alternatives to ensure continued service.
This is important for our system as a consequence of the choice to handle route optimization within the system.

\subsection*{Criteria}
\begin{itemize}[noitemsep]
    \item \textit{Performance}: Rapid detection and response to disruptions.
    \item \textit{Usability}: Clear and timely communication with passengers.
    \item \textit{Functionality}: Provision of alternative transport options.
    \item \textit{Integrability}: Integration with existing operational and communication systems.
    \item Minimization of negative impact on passenger experience.
    \item Compliance with safety and regulatory standards.
\end{itemize}


\subsection*{Option 1: Real-Time Alert System}
\subsubsection*{Pros}
\begin{itemize}
    \item Quickly informs passengers about disruptions.
\end{itemize}
\subsubsection*{Cons}
\begin{itemize}
    \item Requires passengers to actively seek updates.
\end{itemize}

\subsection*{Option 2: Manual Intervention Protocol}
\subsubsection*{Pros}
\begin{itemize}
    \item Personalized assistance to passengers for rebooking and advice.
\end{itemize}
\subsubsection*{Cons}
\begin{itemize}
    \item Complex to implement and integrate with existing systems.
\end{itemize}

\subsection*{Option 3: Threshold-Based Re-Optimization}
\subsubsection*{Pros}
\begin{itemize}
    \item Focuses on significant disruptions, optimizing system and passenger resources.
    \item Reduces the number of unnecessary passenger notifications for minor issues.
\end{itemize}
\subsubsection*{Cons}
\begin{itemize}
    \item Minor delays may not trigger system responses, potentially accumulating unnoticed.
    \item Terminal-based information may not effectively reach all passengers.
\end{itemize}

\subsection*{Option 4: Continuous Optimization}
\subsubsection*{Pros}
\begin{itemize}
    \item Offers constant adaptability to real-time conditions, potentially enhancing system responsiveness.
\end{itemize}
\subsubsection*{Cons}
\begin{itemize}
    \item Could result in frequent, possibly confusing updates for passengers.
    \item May demand significant computational resources, affecting system efficiency.
\end{itemize}


\subsection*{Decision}
We choose Option 3: Threshold-Based Re-Optimization for its strategic focus on significant disruptions. This decision is made recognizing that while minor disruptions may be overlooked, the emphasis on substantial delays aligns with our goal of efficiently managing resources and maintaining a high-quality passenger experience.

\subsection*{Positive Consequences}
\begin{itemize}
    \item Efficient resource allocation by focusing on significant disruptions ensures the system's responsiveness to passenger needs during major incidents.
    \item Reduction in passenger notification fatigue by limiting communications to significant events, thereby enhancing the relevance and impact of messages received.
    \item Improved system performance and cost efficiency by avoiding unnecessary re-optimization processes for minor disruptions.
\end{itemize}

\subsection*{Negative Consequences}
\begin{itemize}
    \item Potential for minor disruptions to accumulate and impact passenger experience if they are not addressed due to falling below the set threshold.
    \item Risk of insufficient communication if passengers are not near terminals and thus may miss critical information about disruptions and re-optimizations.
    \item Challenges in setting an appropriate threshold that accurately distinguishes between minor and significant disruptions, requiring continuous evaluation and adjustment.
\end{itemize}
